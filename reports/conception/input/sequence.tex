\newpage
\section{Diagrammes de séquence}
	Afin de rendre les choses plus claires, et aussi pour respecter les consignes, différents diagrammes de séquence vont être présentés au lecteur.
	Devant leur taille très volumineuse, ceux ci se trouvent dans le .zip du rapport.

	\subsection{Création d'une nouvelle partie}
		Afin d'éclaicir le rôle de chaque classe, le diagramme de séquence 1 illustre les intéractions entre l'utilisateur et les différentes classes pour créer une nouvelle partie, contenant deux joueurs sur une petite carte.


	\subsection{Déroulement d'un tour et d'une partie}
		Une partie très simple est montrée sur le diagramme de séquence 2. Elle met en jeu Gandalf contre Harry, qui possèdent chacun une unité. Ainsi, le premier à perdre une unité sera vaincu, et sera déclaré plus mauvais magicien d'Asgard\footnote{Les combats étants basés en partie sur la chance, il s'agit d'une forme très avancée de pile ou face.}.




	