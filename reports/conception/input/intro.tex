\section{Introduction}
	Le projet SmallWorld, application directe du cours de Programation Orientée Objet, a pour but de faire réaliser aux étudiants\footnote{De quatrième année du département informatique de l'Insa de Rennes en France.} un jeu 2D au cours du semestre.
	Celui est inspiré de deux jeux, SmallWorld (jeu de plateau), et Civilization (jeu vidéo).
	Tout les deux comportent une carte découpée en cases, sur lesquelles évolueront les unités des différentes factions en tour par tour.

	Ce rapport présente la conception du projet, réprésentée le plus souvent à l'aide de diagrammes UML. Bien que ce rapport soit en français, les diagrammes présentés (qui reflètent le code du projet) contiendront des labels en anglais.

\section{Présentation du jeu}
	Dans le jeu que nous allons réaliser, plusieurs factions (controlée par les joueurs) devront s'affronter sur une seule et même carte. Les factions établies dans les consignes sont les nains, les elfes et les orcs, chacunes avec leurs propres spécificités\footnote{D'autres factions seront potentiellements rajoutées lors de la réalisation du projet, en fonction de l'avancement de celui ci. Cela permettrait notamment d'avoir plus de joueurs dans la partie.}.

	En début de partie, chaque joueur possèdera un certain nombre d'unités, qu'il devra placer sur la carte. Par la suite, ces unités pourront se déplacer sur les cases proches (en fonction de ses points de déplacement restants).
	Lorsqu'une unité voudra se déplacer sur une case occupée par une unité d'une faction adverse, un combat aura lieu. \`A l'issue du combat (victoire, défaite, status quo), l'unité pourra se déplacer ou non sur la case convoitée.

	La partie se termine lorsqu'il ne reste plus qu'un seul joueur possédant au moins une unité.















